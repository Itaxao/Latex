\documentclass[12pt,a4paper]{article}

\PassOptionsToPackage{pdftex}{graphicx}
\usepackage{graphicx}
\usepackage[sfdefault,condensed]{cabin}
\usepackage[T1]{fontenc}
\usepackage[utf8]{inputenc}
\usepackage[brazilian]{babel}
\usepackage{epstopdf}
\usepackage{soul}
\usepackage{array}
\usepackage[table]{xcolor}
\usepackage{lipsum}
\usepackage{subfig}
\RequirePackage[numbers]{natbib}
\usepackage{amsmath}
\usepackage{amssymb}
\usepackage{amsthm}
\usepackage{enumitem}

\newcommand{\CEFETdisciplina}{Contexto Social e Profissional da Engenharia da Computação}
\newcommand{\CEFETprofessor}{Eduardo Miranda}

% aumentando as margens originais do documento
\usepackage{microtype}
\addtolength{\textwidth}{2cm}
\addtolength{\hoffset}{-1cm}
\addtolength{\textheight}{-2cm}

%Para algoritmos ----------------------
\usepackage{hyperref}
\usepackage{listings}
\usepackage{color}
\definecolor{dkgreen}{rgb}{0,0.6,0}
\definecolor{gray}{rgb}{0.5,0.5,0.5}
\definecolor{mauve}{rgb}{0.58,0,0.82}
\definecolor{bg}{rgb}{0.95,0.95,0.95}
\lstset{
  language=Python,
  basicstyle=\footnotesize,
  numbers=left,
  numberstyle=\tiny\color{gray},
  stepnumber=1,
  numbersep=5pt,
  showspaces=false,
  showstringspaces=false,
  showtabs=false,
  frame=single,
  rulecolor=\color{black},
  tabsize=4,
  captionpos=b,
  breaklines=true,
  breakatwhitespace=false,
  title=\lstname,
  keywordstyle=\color{blue},
  commentstyle=\color{dkgreen},
  stringstyle=\color{mauve}
}
%--------------------------------------

\usepackage{lastpage}

% paragrafos
\setlength{\parskip}{.5cm}
\setlength{\parindent}{0pt}

\usepackage{fancyhdr}
\pagestyle{fancy}
\fancypagestyle{firstpage}{%
  \fancyhf{}%
  \renewcommand{\headheight}{85pt}
  \fancyhead[L]{\includegraphics[width=3.6cm]{images/logo.png}\vspace{.3cm}}
  \fancyhead[R]{%
        \textbf{CENTRO FEDERAL DE EDUCAÇÃO TECNOLÓGICA DE
        MINAS GERAIS}\\CAMPUS V - UNIDADE DIVINÓPOLIS - Engenharia
        da Computação\\\vspace{.3cm}\CEFETdisciplina\\\CEFETprofessor\\\vspace{-.3cm}
  }%
}
\fancypagestyle{otherpage}{%
  \fancyhf{}%
  \renewcommand{\headheight}{45pt}%
  \renewcommand{\footrulewidth}{1pt}
  \renewcommand{\headrulewidth}{1pt}
  \fancyhead[R]{\CEFETdisciplina\\
    \CEFETprofessor\\\vspace{-.3cm}
  }%
  \addtolength{\textheight}{50pt}
}

% Enables use of commands like \@author
\makeatletter
\newcommand{\CEFETtitulo}[1]{
\begin{center}
\begin{Large}
\textbf{\\#1\\}
\end{Large}
\vspace{.5cm}
{\large \textbf{\@author}} % This now correctly displays the author's name
\end{center}
\vspace{.5cm}
}
\makeatother % Reverts @ to its normal category code

\newcommand{\CEFETsubtitulo}[1]{
\begin{center}
\begin{large}
\textbf{#1\\}
\end{large}
\end{center}
}

\newcommand{\CEFETopen}{
    \selectlanguage{brazilian}
    \pagestyle{otherpage}
    \thispagestyle{firstpage}
}

%para colunas--------------------------
\usepackage{multicol}
\setlength{\columnsep}{1cm}
\definecolor{cefetblue}{RGB}{2, 65, 112}
\makeatletter
\def\columnseprulecolor\vrule\@width\columnseprule{
\vbox to \ht\mult@rightbox{\leaders\vbox{\kern2pt\kern2pt}\vfill}}
\makeatother
\setlength{\columnseprule}{5px}

\author{*********} % This sets the author for the document
%--------------------------------------
\begin{document}
    \title{Modelo Relatório CEFET} % This is still needed for \@title and internal LaTeX use
    \CEFETopen
    \CEFETtitulo{Inovações Tecnlógicas e Futuro da Engenharia da Computação}

    \section{Introdução}
      O tema do nosso grupo é “Inovações Tecnológicas e o Futuro da Engenharia da Computação”. Neste trabalho, vamos abordar áreas como Inteligência Artificial, Computação Quântica, Internet das Coisas, Cidades Inteligentes e outras tendências emergentes que têm transformado o papel do engenheiro da computação na sociedade.
    Para começar, é importante destacar que a área da computação está em constante evolução. A cada ano, novas ferramentas, técnicas e tecnologias surgem, tornando obsoletas soluções antes consideradas modernas. A computação não apenas evolui — ela se reinventa constantemente.
    Um bom exemplo disso é a popularização da Inteligência Artificial. Nos últimos anos, vimos o surgimento de diversas IAs na internet, que passaram de sistemas simples para ferramentas capazes de gerar textos, imagens e vídeos realistas. Basta comparar exemplos antigos — como o famoso vídeo do Will Smith comendo macarrão, mal renderizado — com os conteúdos gerados hoje, que são muitas vezes tão realistas que podem confundir até o olhar humano.
    E essa revolução tecnológica não se limita ao campo acadêmico. Há cerca de 10 anos, discutíamos o impacto das redes sociais como o Instagram, criado em 2010. Hoje, os debates giram em torno de computação quântica, IA generativa e cidades conectadas. Isso mostra como a computação influencia diretamente a sociedade e está presente em todas as áreas.
    Ferramentas antigas foram sendo substituídas por soluções modernas: os disquetes deram lugar ao Google Drive e ao Dropbox; os modens barulhentos foram superados por internet de fibra óptica, 5G e Wi-Fi ultrarrápido. Hoje, muitos desses objetos estão em museus, como marcos de uma era que passou rapidamente.
    Essas mudanças impactaram diversas áreas da sociedade:
    \begin{enumerate}[label=\Roman*]
      \item Comunicação: De cartas e e-mails, passamos para WhatsApp e redes sociais com mensagens em tempo real.
      \item Trabalho: Do ambiente físico cheio de arquivos (como vi pessoalmente quando trabalhei na Poupa Minas), migramos para o home office e armazenamento em nuvem.
      \item	Consumo: Antes limitado a lojas físicas e dinheiro em espécie, hoje temos e-commerce, QR Code e Pix.
      \item Educação: De aulas presenciais e livros físicos, passamos para o ensino a distância, plataformas digitais e cursos online.
      \item Saúde: De consultas presenciais a atendimentos por telemedicina, inclusive com médicos de outras cidades ou estados.
      \item obilidade urbana: Substituímos os táxis e mapas impressos por aplicativos como Uber, Waze e FaleBus, que informam em tempo real a localização dos ônibus.
    \end{enumerate}

    Com tudo isso, é evidente que a computação está presente em todas as áreas da vida — não apenas em empresas e laboratórios. Ela está redesenhando a sociedade.
    Nesse cenário, a Engenharia da Computação também mudou. Ela deixou de ser uma área focada apenas em "mexer com computador" ou escrever códigos. Hoje, é uma disciplina multidisciplinar que integra hardware, software, inteligência artificial, ciência de dados, automação, robótica e design de sistemas inteligentes.
    O engenheiro de computação atual pode atuar em diversos setores e deve estar preparado para aprender continuamente, adaptar-se às novas tecnologias e solucionar problemas complexos. Mais do que criar sistemas, ele deve refletir sobre os impactos que suas criações terão na sociedade.
    Um bom exemplo é o engenheiro que desenvolve um algoritmo de IA para detectar doenças, projeta sensores para cidades inteligentes ou otimiza o uso de energia em uma fábrica usando computação em nuvem.
    Discutir inovação tecnológica é fundamental porque essas mudanças fazem parte do nosso dia a dia e moldam nossas vidas. Refletir sobre elas é essencial para nos prepararmos para o futuro — um futuro que chega cada vez mais rápido. As inovações podem trazer benefícios imensos, mas também desafios éticos e sociais. Por isso, é importante analisar tanto os impactos positivos quanto os negativos.
    E, por fim, o profissional que deseja atuar nesse futuro precisa estar sempre atualizado, estudando as tendências e entendendo o contexto em que a tecnologia está inserida.
    Agora, vou apresentar brevemente os temas que meus colegas irão aprofundar:
    \begin{enumerate}[label=\Roman*]
      \item Inteligência Artificial (IA): Criação de sistemas capazes de aprender, tomar decisões e imitar o raciocínio humano. Já está presente em assistentes virtuais, recomendações de conteúdo e geração de imagens e textos.
      \item Computação Quântica: Tecnologia emergente que utiliza qubits em vez de bits. Permite resolver problemas extremamente complexos de forma rápida e tem grande potencial em áreas como criptografia, medicina e simulações.
      \item Cidades Inteligentes: Uso de tecnologias e dados para melhorar a vida urbana, com soluções como semáforos inteligentes, gestão de tráfego, iluminação automatizada e monitoramento ambiental.
    \end{enumerate}

    \section{IA e Machine Learning}

   \subsection{Desafios relacionados às IA's}

      As IAs apresentam uma série de desafios que vão além das questões éticas e legais. Um dos principais é a \textbf{complexidade técnica} de seu desenvolvimento e manutenção. 
    Modelos de IA avançados exigem grandes volumes de dados de alta qualidade e poder computacional significativo, o que limita o acesso a essas tecnologias para muitas 
    organizações. Além disso, a \textbf{interpretabilidade} dos modelos de IA, especialmente os mais complexos como as redes neurais profundas, é um desafio constante. 
    Entender como e por que uma IA toma determinadas decisões é crucial para garantir a confiabilidade e a responsabilidade de seu uso, especialmente em áreas críticas 
    como saúde ou justiça.Outro ponto de preocupação é a \textbf{segurança cibernética} das IAs. Sistemas de IA podem ser vulneráveis a ataques que visam manipular seus 
    dados de treinamento, induzir erros ou até mesmo paralisar seu funcionamento. A garantia da robustez e da resiliência desses sistemas é fundamental para evitar usos 
    maliciosos e garantir a integridade das informações. A \textbf{dependência excessiva} da IA em certas áreas também pode gerar problemas, como a perda de habilidades 
    humanas e a fragilização de sistemas caso as IAs falhem ou sejam comprometidas.

    \subsection{Desafios éticos em relação às IA's}

      Segundo a revista Complexitas(2019)\citet{complexitas6633} é de suma importância um interesse tanto acadêmico quanto governamental ao que tange o 
    problema relacionado a forma como a IA e as empresas detentoras de seus domínios utilizam os dados de seus usuários, mas também a forma como será 
    punido àquele utiliza-la de maneira não ética, para isso é necessário formulação de leis de maneiras concisas e funcionais, baseando-se não somente no
    usuário, mas também na forma como a empresa detentora dos direitos pode e deverá ser punida, contudo, é complexo a funcionalidade dessas leis,
    principalmente levando em conta a quantidade de diferentes domínios utilizados para apenas um modelo de linguagem generativa, a complexidade dessas leis
    é bem crível quando pensamos de maneira objetiva, imagina que um usuário brasileiro em um voo internacional em um avião de uma empresa italiana, 
    passando por portugal, e essa pessoa comete um crime utilizando o modelo de linguagem generativa de uma empresa estadunidense, a burocracia relacionada
    ao caso será destinada a quem, Brasil?Italia?Estados Unidos ou até mesmo Portugal? Levando em conta as Leis atuais não é possível haver um consenso
    quanto o julgamento deste caso, quem será punido é o criminoso ou a empresa da "IA" que permitiu que ele fizesse isso? Portanto é necessário uma análise
    ética a forma como será utilizada e quem será punido nesse e outros casos parecidos.

      Ademais temos questões éticas mais discutidas, como a importância da privacidade relacionada aos usuários e também à forma como os dados recolhidos
    pela IA serão utilizados, dessa forma podemos observar não apenas as questões éticas legais como também as sociais que são apresentadas diante de um tema tão complexo,
    para resolver esses problemas citados — além de outros — foi criado uma convenção chamada IA responsável, não apenas para mitigar os problemas atuais, 
    mas para também evitar problemas futuros e possivelmente maiores, essa convenção/regra é utilizada agora por todos desenvolvedores éticos que buscam melhorar
    a vida humana como todo e também evoluir tecnlogicamente de forma segura, essa ideia é defendida por diversos pesquisadores e utiliza de algumas regras,como vemos no \cite{Almeida_Nas_2024}. essas sendo:
    \begin{enumerate} [label=\Roman*]
      \item Legitimidade e Competência;
      \item Minimização de Danos;
      \item Segurança e Privacidade;
      \item Transparência;
      \item Interpretabilidade e Explicabilidade;
      \item Manutenibilidade;
      \item Contestabilidade e Auditabilidade;
      \item Responsabilidade;
      \item Limitação de Impactos ambientais;
    \end{enumerate}

    \subsection{Impactos no mercado de trabalho}

      A ascensão das IAs está remodelando o mercado de trabalho de maneira profunda e multifacetada. Por um lado, há uma preocupação generalizada com a 
    \textbf{automação de tarefas}, o que pode levar à substituição de empregos em setores como manufatura, atendimento ao cliente e até mesmo em algumas áreas administrativas.
    Trabalhos rotineiros e repetitivos são os mais suscetíveis à automação, exigindo que a força de trabalho se adapte e adquira novas habilidades.Por outro lado, a IA 
    também está \textbf{criando novas oportunidades de trabalho} e aprimorando funções existentes. A demanda por especialistas em IA, cientistas de dados, engenheiros 
    de machine learning e profissionais de ética em IA está em ascensão. Além disso, a IA pode \textbf{aumentar a produtividade} e a eficiência em diversas indústrias,
    permitindo que os trabalhadores se concentrem em tarefas mais criativas, estratégicas e complexas. A colaboração entre humanos e IAs (chamada de "inteligência aumentada")
    é uma tendência crescente, onde a IA atua como uma ferramenta para potencializar as capacidades humanas, e não para substituí-las por completo. No entanto, é crucial 
    investir em \textbf{requalificação e educação} para que a força de trabalho possa se adaptar a essas mudanças e aproveitar as novas oportunidades que surgirão.

    \subsection{Tendências Futuras}

      O futuro das IAs promete avanços significativos e transformações ainda mais profundas na sociedade. Uma das principais tendências é o desenvolvimento 
    de \textbf{IA Generativa mais avançada}, capaz de criar conteúdos cada vez mais sofisticados, como textos, imagens, músicas e até mesmo códigos de programação, 
    com maior autonomia e criatividade. Isso terá implicações vastas para indústrias criativas, desenvolvimento de software e comunicação.Outra área de crescimento 
    será a \textbf{IA explicável (XAI)}. À medida que as IAs se tornam mais complexas e são aplicadas em áreas críticas, a capacidade de compreender e interpretar suas 
    decisões será crucial. A XAI busca desenvolver métodos e técnicas para tornar os modelos de IA mais transparentes e compreensíveis para os humanos, aumentando a 
    confiança e a responsabilidade em seu uso. Além disso, a \textbf{IA incorporada e pervasiva} será cada vez mais presente em nosso cotidiano, com dispositivos inteligentes e sistemas autônomos integrados em ambientes residenciais, urbanos e industriais.
    A pesquisa em \textbf{IA de baixo consumo de energia} também é uma tendência importante, visando reduzir o impacto ambiental dos grandes modelos de IA e torná-los 
    mais acessíveis. A \textbf{regulamentação e governança da IA} continuarão a ser um tema central, com a formulação de leis e diretrizes para garantir o uso ético e 
    responsável da tecnologia em escala global. Finalmente, a \textbf{colaboração humano-IA} se aprofundará, com sistemas de IA se tornando parceiros mais eficazes para 
    a resolução de problemas complexos e a inovação.

    \section{Computação Quântica e Blockchain} 
      A Computação Quântica representa um novo paradigma de processamento de informações, baseado em princípios da mecânica quântica, como superposição, entrelaçamento e 
    interferência. Em vez de bits clássicos (0 ou 1), os computadores quânticos utilizam qubits, que podem assumir múltiplos estados simultaneamente. Isso permite resolver
    certos problemas muito mais rapidamente do que os computadores tradicionais, como é o caso do algoritmo de Shor, capaz de fatorar grandes números inteiros de forma eficiente.
    Já o Blockchain é uma tecnologia descentralizada de registro digital, utilizada principalmente para registrar transações de maneira segura, transparente e imutável.
    Sua segurança é garantida por métodos criptográficos, como a criptografia assimétrica (ex: RSA, ECC) e funções de hash, que dependem da dificuldade computacional de certos problemas matemáticos.
    Contudo, a crescente evolução da computação quântica representa uma ameaça à segurança dos atuais sistemas de blockchain. Algoritmos quânticos como o de Shor poderiam,
    futuramente, quebrar os sistemas de criptografia utilizados para proteger as chaves privadas e as assinaturas digitais, possibilitando ataques a carteiras e transações.
      Além disso, o algoritmo de Grover também acelera buscas em sistemas de chave simétrica, reduzindo a resistência de protocolos como o AES. Como resposta a esses riscos, 
    surgem soluções de criptografia pós-quântica (PQC), com algoritmos resistentes a ataques de computadores quânticos, 
    e propostas para atualizar as infraestruturas atuais de blockchain. Algumas abordagens incluem o uso de assinaturas baseadas em hash, criptografia de reticulado e até 
    blockchains quânticas, que utilizariam canais quânticos (como QKD – distribuição quântica de chaves) para garantir segurança de comunicação entre nós da rede.
    Além dos desafios, essa convergência abre novas possibilidades: blockchains quânticas poderiam utilizar entrelaçamento quântico para verificar a ordem de blocos ou executar contratos inteligentes de maneira paralela e segura. Apesar disso, a implementação prática ainda enfrenta limitações, como a escalabilidade do hardware quântico e a padronização de algoritmos resistentes.
      Concluindo, a interseção entre Computação Quântica e Blockchain representa tanto um desafio técnico urgente, quanto uma oportunidade para inovação disruptiva.
    A preparação para esse futuro exige atualização constante, investimento em pesquisa e adoção de novos padrões de segurança. Como estudante e entusiasta de tecnologia,
     acredito que compreender essa relação é fundamental para antecipar mudanças e contribuir com soluções seguras e eficazes.




    \subsection{}

\clearpage
\bibliographystyle{plainnat}
\bibliography{referencias.bib}
\end{document}