\documentclass[12pt,a4paper]{article}

\PassOptionsToPackage{pdftex}{graphicx}
\usepackage{graphicx}
\usepackage[sfdefault,condensed]{cabin}
\usepackage[T1]{fontenc}
\usepackage[utf8]{inputenc}
\usepackage[brazilian]{babel}
\usepackage{epstopdf}
\usepackage{soul}
\usepackage{array}
\usepackage[table]{xcolor}
\usepackage{lipsum}
\usepackage{subfig}
\RequirePackage[numbers]{natbib}
\usepackage{amsmath}
\usepackage{amssymb}
\usepackage{amsthm}

\newcommand{\CEFETdisciplina}{Contexto Social e Profissional da Engenharia da Computação}
\newcommand{\CEFETprofessor}{Eduardo Miranda}

% aumentando as margens originais do documento
\usepackage{microtype}
\addtolength{\textwidth}{2cm}
\addtolength{\hoffset}{-1cm}
\addtolength{\textheight}{-2cm}

%Para algoritmos ----------------------
\usepackage{hyperref}
\usepackage{listings}
\usepackage{color}
\definecolor{dkgreen}{rgb}{0,0.6,0}
\definecolor{gray}{rgb}{0.5,0.5,0.5}
\definecolor{mauve}{rgb}{0.58,0,0.82}
\definecolor{bg}{rgb}{0.95,0.95,0.95}
\lstset{
  language=Python,
  basicstyle=\footnotesize,
  numbers=left,
  numberstyle=\tiny\color{gray},
  stepnumber=1,
  numbersep=5pt,
  showspaces=false,
  showstringspaces=false,
  showtabs=false,
  frame=single,
  rulecolor=\color{black},
  tabsize=4,
  captionpos=b,
  breaklines=true,
  breakatwhitespace=false,
  title=\lstname,
  keywordstyle=\color{blue},
  commentstyle=\color{dkgreen},
  stringstyle=\color{mauve}
}
%--------------------------------------

\usepackage{lastpage}

% paragrafos
\setlength{\parskip}{.5cm}
\setlength{\parindent}{0pt}

\usepackage{fancyhdr}
\pagestyle{fancy}
\fancypagestyle{firstpage}{%
  \fancyhf{}%
  \renewcommand{\headheight}{85pt}%
  \fancyhead[L]{\includegraphics[width=3.6cm]{images/logo.png}\vspace{.3cm}}
  \fancyhead[R]{%
        \textbf{CENTRO FEDERAL DE EDUCAÇÃO TECNOLÓGICA DE
        MINAS GERAIS}\\CAMPUS V - UNIDADE DIVINÓPOLIS - Engenharia
        da Computação\\\vspace{.3cm}\CEFETdisciplina\\\CEFETprofessor\\\vspace{-.3cm}
  }%
}
\fancypagestyle{otherpage}{%
  \fancyhf{}%
  \renewcommand{\headheight}{45pt}%
  \renewcommand{\footrulewidth}{1pt}
  \renewcommand{\headrulewidth}{1pt}
  \fancyhead[R]{\CEFETdisciplina\\
    \CEFETprofessor\\\vspace{-.3cm}
  }%
  \addtolength{\textheight}{50pt}
}

% Enables use of commands like \@author
\makeatletter
\newcommand{\CEFETtitulo}[1]{
\begin{center}
\begin{Large}
\textbf{\\#1\\}
\end{Large}
\vspace{.5cm}
{\large \textbf{\@author}} % This now correctly displays the author's name
\end{center}
\vspace{.5cm}
}
\makeatother % Reverts @ to its normal category code

\newcommand{\CEFETsubtitulo}[1]{
\begin{center}
\begin{large}
\textbf{#1\\}
\end{large}
\end{center}
}

\newcommand{\CEFETopen}{
    \selectlanguage{brazilian}
    \pagestyle{otherpage}
    \thispagestyle{firstpage}
}

%para colunas--------------------------
\usepackage{multicol}
\setlength{\columnsep}{1cm}
\definecolor{cefetblue}{RGB}{2, 65, 112}
\makeatletter
\def\columnseprulecolor\vrule\@width\columnseprule{
\vbox to \ht\mult@rightbox{\leaders\vbox{\kern2pt\kern2pt}\vfill}}
\makeatother
\setlength{\columnseprule}{5px}

\author{*********} % This sets the author for the document
%--------------------------------------
\begin{document}
    \title{Modelo Relatório CEFET} % This is still needed for \@title and internal LaTeX use
    \CEFETopen
    \CEFETtitulo{Inovações Tecnlógicas e Futuro da Engenharia da Computação}


    \section{IA e Machine Learning}


    \subsection{Desafios éticos em relação às IA's}

    Segundo a revista Complexitas(2019)\citet{complexitas6633} é de suma importância um interesse tanto acadêmico quanto governamental ao que tange o 
    problema relacionado a forma como a IA e as empresas detentoras de seus domínios utilizam os dados de seus usuários, mas também a forma como será 
    punido àquele utiliza-la de maneira não ética, para isso é necessário formulação de leis de maneiras concisas e funcionais, baseando-se não somente no
    usuário, mas também na forma como a empresa detentora dos direitos pode e deverá ser punida, contudo, é complexo a funcionalidade dessas leis,
    principalmente levando em conta a quantidade de diferentes domínios utilizados para apenas um modelo de linguagem generativa, a complexidade dessas leis
    é bem crível quando pensamos de maneira objetiva, imagina que um usuário brasileiro em um voo internacional em um avião de uma empresa italiana, 
    passando por portugal, e essa pessoa comete um crime utilizando o modelo de linguagem generativa de uma empresa estadunidense, a burocracia relacionada
    ao caso será destinada a quem, Brasil?Italia?Estados Unidos ou até mesmo Portugal? Levando em conta as Leis atuais não é possível haver um consenso
    quanto o julgamento deste caso, quem será punido é o criminoso ou a empresa da "IA" que permitiu que ele fizesse isso? Portanto é necessário uma análise
    ética a forma como será utilizada e quem será punido nesse e outros casos parecidos.







\clearpage
\bibliographystyle{plainnat}
\bibliography{referencias}
\end{document}